\documentclass{beamer}

\mode<presentation>{
\usetheme{Dresden}
\setbeamercovered{transparent}
\usecolortheme{lsc}
}

\mode<handout>{
  % tema simples para ser impresso
  \usepackage[bar]{beamerthemetree}
  % Colocando um fundo cinza quando for gerar transparências para serem impressas
  % mais de uma transparência por página
  \beamertemplatesolidbackgroundcolor{black!5}
}

\usepackage{amsmath,amssymb}
\usepackage[brazil]{varioref}
\usepackage[english,brazil]{babel}
\usepackage[utf8]{inputenc}
%\usepackage[latin1]{inputenc}
\usepackage{graphicx}
\usepackage{listings}
\usepackage{url}
\usepackage{colortbl}

\beamertemplatetransparentcovereddynamic

%https://tex.stackexchange.com/questions/100838/beamer-dresden-theme-miniframes-appeareance-and-frame-number-insertion
%add number of pages - START
\newcommand{\frameofframes}{/}
\newcommand{\setframeofframes}[1]{\renewcommand{\frameofframes}{#1}}

\setframeofframes{de}
\makeatletter
\setbeamertemplate{footline}
  {%
    \begin{beamercolorbox}[colsep=1.5pt]{upper separation line foot}
    \end{beamercolorbox}
    \begin{beamercolorbox}[ht=2.5ex,dp=1.125ex,%
      leftskip=.3cm,rightskip=.3cm plus1fil]{author in head/foot}%
      \leavevmode{\usebeamerfont{author in head/foot}\insertshortauthor}%
      \hfill%
      {\usebeamerfont{institute in head/foot}\usebeamercolor[fg]{institute in head/foot}\insertshortinstitute}%
    \end{beamercolorbox}%
    \begin{beamercolorbox}[ht=2.5ex,dp=1.125ex,%
      leftskip=.3cm,rightskip=.3cm plus1fil]{title in head/foot}%
      {\usebeamerfont{title in head/foot}\insertshorttitle}%
      \hfill%
      {\usebeamerfont{frame number}\usebeamercolor[fg]{frame number}\insertframenumber~\frameofframes~\inserttotalframenumber}
    \end{beamercolorbox}%
    \begin{beamercolorbox}[colsep=1.5pt]{lower separation line foot}
    \end{beamercolorbox}
  }
\makeatother
%add number of pages - END

\title[Análise de desempenho de aplicações java utilizando programação reativa sem bloqueio e tradicional com bloqueio
]{\normalsize Análise de desempenho de aplicações java utilizando programação reativa sem bloqueio e tradicional com bloqueio
}
\author[Rafael Elias Ribeiro Barbosa]{%
  Rafael Elias Ribeiro Barbosa\\
  Orientador: D.Sc. Marum Simão Filho}
  \institute[UNI7]{
  %
     Especialização em Arquitetura, Design e Implem. de Sistemas para Internet\\
     Centro Universitário 7 de Setembro}

% Se comentar a linha abaixo, irá aparecer a data quando foi compilada a apresentação  
\date{10 de Outubro de 2018}

\pgfdeclareimage[height=1cm]{inf}{figs/uni7-logo.png}

% pode-se colocar o LOGO assim
\logo{\pgfuseimage{inf}}

\AtBeginSection[]{
  \begin{frame}<beamer>
    \frametitle{Roteiro}
    \tableofcontents[currentsection,currentsubsection]
  \end{frame}
}

\begin{document}

\begin{frame}
\titlepage
\end{frame}

\begin{frame}
\frametitle{Roteiro}
\tableofcontents
\end{frame}


\section{Introdução}
\frame{
    \frametitle{Introdução}
      \begin{itemize}
         \item Otimizar a utilização do \textit{hardware}
         \item Programção reativa
         
      \end{itemize}
}

\frame{
    \frametitle{Justificativa e Objetivos}
    \begin{itemize}   
       \item Justificativa
       \begin{itemize}
           \item Suporte nativo a programação reativa: Java 9 e Spring 5
       \end{itemize}
       
       \item Objetivos
       \begin{itemize}
       	   \item Geral
       	   \begin{itemize}    
       	        \item Comparação entre dois paradigmas de programação, em relação à performance
       	   \end{itemize}  
       	   
       	   \item Específicos 
       	   \begin{itemize} 
       	        \item Discutir os conceitos de programação reativa
       	        \item Fornecer exemplo de implementação
       	        \item Analisar a performance das aplicações
       	   \end{itemize}  
       \end{itemize}
   \end{itemize}
}


\section{Fundamentação}

\pgfdeclareimage[height=2.5cm]{depGreph}{figs/dependency-graph.png}%
\frame{
	\frametitle{Programação Reativa}
	\begin{itemize}
	    \item Paradigma declarativo
        \item \textit{o que fazer} e \textit{quando fazer}
	\end{itemize}
	\begin{columns}
        \column{4cm}
            {\fontfamily{qcr}\selectfont
                \textbf{int} x = 1; \\
            	\textbf{int} y = 2; \\
            	\textbf{int} sum = x + y;
            }
		\column{5cm}
		    \\
			\pgfuseimage{depGreph}
			\center
			\begin{tiny}
                Fonte: Bainomugisha et al. (2013)
            \end{tiny}
    \end{columns}
}

\frame{
	\frametitle{Modelos de Avalição (Evaluation model)}
	\begin{itemize}
		\item Quem é o responsável pela a propagação das alterações?
		\item Modelo \textit{push}
		\item Modelo \textit{pull}
		
	\end{itemize}
}

\frame{
	\frametitle{Reactive Streams}
	\begin{itemize}
		\item Especificação para implementações que estejam em conformidade com os princípios da programação reativa
		    \begin{itemize}
		        \item fluxos assíncronos com \textit{back pressure non-blocking}
		    \end{itemize}
		\item Modelo \textit{push}
		\item Java 9 \textbf{java.util.current.Flow}
		\item Project Reactor (Spring 5)
		
	\end{itemize}
}

\pgfdeclareimage[height=2.5cm]{rectTraits}{figs/reactive-traits.png}%
\frame{
	\frametitle{Manifesto Reativo}
	\begin{itemize}
		\item Propor uma arquitetura de software que seja capaz de atender as demandas dos sistemas atuais
		\item Sistemas reativos
	\end{itemize}
	\center
	\pgfuseimage{rectTraits}
	\center
	\begin{tiny}
        Fonte: Bonér et al. (2018)
    \end{tiny}
}

\section{Desenvolvimento}
\frame{
    \frametitle{Aplicações}
    \begin{columns}
        \column{4cm}
            \begin{itemize}
        		\item \textbf{Reativa:} \\
        		- Java 8 \\
		        - Spring 5 WebFlux \\
		        - Netty
        	\end{itemize}
        \column{4cm}
            \begin{itemize}
        		\item \textbf{Tradicional:} \\
        		- Java 8 \\
		        - Spring 5 MVC \\
		        - Tomcat
        	\end{itemize}
    \end{columns}
}

\pgfdeclareimage[height=1cm]{mongo}{figs/mongoDB.png}%
\pgfdeclareimage[height=1.5cm]{mysql}{figs/mysql.png}%
\frame{
    \frametitle{Bancos de dados}
        \begin{columns}
            \column{6.5cm}
                \begin{itemize}
            		\item Reativa:
        	        \pgfuseimage{mongo}
        	        
            		\item Tradicional:
        	        \pgfuseimage{mysql}
            	\end{itemize}
        	
        	\column{3cm}
            	\begin{table}
        		\renewcommand{\figurename}{Table}
        		\centering
        		\begin{tabular}{|l|}
        		    \hline
        		    \multicolumn{1}{|c|}{\textbf{User}} \\
        			\hline 
        			    id: Long \\ 
        			\hline 
        			    name: String \\ 
        			\hline 
        			    age: Long \\
        			\hline
        		\end{tabular}
        	\end{table}
    	\end{columns}
}

\pgfdeclareimage[height=1.3cm]{docker}{figs/docker.png}%
\frame{
    \frametitle{Ambiente de simulação}
    \pgfuseimage{docker} \\
    \begin{itemize}
        \item Processador Intel Core i7-4790 com 3.60GHz 
		\item Sistema operacional: Boot2Docker 18.06.0-ce \item Tipo do sistema operacional: Linux 
		\item Arquitetura: x86\_64 
		\item CPUs: 4 
		\item Memoria: 1.955GiB 
    \end{itemize}
}

\pgfdeclareimage[height=1.3cm]{gatling}{figs/gatling.jpg}%
\frame{
    \frametitle{Ferramenta de testes}
    
	\pgfuseimage{gatling} \\
    \begin{itemize}
        \item Ferramenta de testes de performance \\ 
    \end{itemize}
}

\section{Resultados e Discussão}
\frame{
	\frametitle{Configuração dos experimentos}
	
	\begin{itemize}
		\item Os resultados foram separados em 2 etapas, considerando o volume de registros:
    		\begin{itemize}
    		    \item 1000 registros
    		    \item 2000 registros
    		\end{itemize}
    		
    	\item Em cada uma das etapas, foram realizadas três cargas de testes:
    	    \begin{itemize}
    		    \item 100 usuários
    		    \item 1000 usuários
    		    \item 2000 usuários
    		\end{itemize}
	\end{itemize}
}

\pgfdeclareimage[height=3.5cm]{1000-100}{figs/chart1000-100u.png}%
\pgfdeclareimage[height=3.5cm]{1000-2000}{figs/chart1000-2000u.png}%
\frame{
	\frametitle{Resultados}
    	\begin{columns}
            \column{\dimexpr\paperwidth-178pt}
                \center
                \begin{footnotesize}
                    1000 registros - 100 usuários
                \end{footnotesize}
                \pgfuseimage{1000-100}
                
            \column{\dimexpr\paperwidth-178pt}
                \center
                \begin{footnotesize}
                     1000 registros - 2000 usuários
                \end{footnotesize}
                \pgfuseimage{1000-2000}
                
    	\end{columns}
}


\pgfdeclareimage[height=3.5cm]{2000-100}{figs/chart2000-100u.png}%
\pgfdeclareimage[height=3.5cm]{2000-2000}{figs/chart2000-2000u.png}%
\frame{
	\frametitle{Resultados (Continuação)}
    	\begin{columns}
            \column{\dimexpr\paperwidth-178pt}
                \center
                \begin{footnotesize}
                    2000 registros - 100 usuários
                \end{footnotesize}
                \pgfuseimage{2000-100}

            \column{\dimexpr\paperwidth-178pt}
                \center
                \begin{footnotesize}
                     2000 registros - 2000 usuários
                \end{footnotesize}
                \pgfuseimage{2000-2000}
                
    	\end{columns}
}

\section{Conclusão}

\pgfdeclareimage[height=5cm]{resumo}{figs/resumo-resultados.png}%
\frame{
   \frametitle{Resumo dos resultados}
   \center
   \pgfuseimage{resumo}
}

\frame{
   \frametitle{Dificuldades encontradas}
   \begin{itemize}
        \item Tradicional
        \begin{itemize}
            \item j.u.c.TimeoutException: Request timeout to /192.168.99.100:8080 after 60000 ms
        \end{itemize}
        
        \item Reativa
        \begin{itemize}
            \item Too many threads are already waiting for a connection. Max number of threads (maxWaitQueueSize) of 500 has been exceeded
        \end{itemize}
   \end{itemize}
}

\logo{\pgfuseimage{inf}}
\frame{
   \frametitle{Trabalhos Futuros}
   	\begin{itemize}
   		\item Realizar outras comparações, dessa vez, considerando aspectos como o consumo de memória, a utilização da CPU e a utilização da rede
        \item Realizar as mesmas comparações mas com o mesmo banco de dados e mesmo servidor, para ambas as aplicações
        \item Verificar o desempenho em relação a aplicações feitas em outras linguagens que também têm suporte à programação reativa, como Javascript
   	\end{itemize}
}


\begin{frame}
\titlepage
\end{frame}

\end{document}